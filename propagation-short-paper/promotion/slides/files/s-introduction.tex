% !TEX root = ../main.tex
%---------------------------------------------------------------------------------------------------
%---------------------------------------------------------------------------------------------------
\section*{Introduction}\addtocounter{framenumber}{-1}
%---------------------------------------------------------------------------------------------------
%---------------------------------------------------------------------------------------------------
\begin{frame}{Structural models}\vspace{0.3cm}

Highly parameterized computational economic models that represent deep structural relationship of theoretical economic models that are invariant to policy changes and are estimated to data \citep{Hood.1953}.\vspace{0.3cm}

\pause
  \heading{Famous examples}\vspace{0.3cm}
	\begin{itemize}\setlength\itemsep{1em}
  \item Optimal replacement of GMC bus engines
	\item The career decisions of young men
  \item The technology of skill formation
	\end{itemize}
\end{frame}

%---------------------------------------------------------------------------------------------------
%---------------------------------------------------------------------------------------------------

\begin{frame}{Structural models}

\begin{multicols}{2}
	\heading{Motivation}\vspace{0.3cm}
	\begin{itemize}\setlength\itemsep{1em}
	\item Facilitate academic rigor
	\item Study mechanisms
	\item Predict public policies
	\end{itemize}

	\pause
  \break
  \heading{Uncertainty}\vspace{0.3cm}
	\begin{itemize}\setlength\itemsep{1em}
	\item Model specification
	\item Numerical approximation\pause
  \item {\color{red} Parameter estimation}
	\end{itemize}
	\end{multicols}


\end{frame}
%---------------------------------------------------------------------------------------------------
%---------------------------------------------------------------------------------------------------
\begin{frame}{Coping with uncertainty}\vspace{0.3cm}

A proper accounting of uncertainty is a prerequisite for using computational models in policy-making \citep{Council.2012,SAPEA.2019}.\vspace{0.3cm}

  \pause

  \begin{multicols}{2}
  	\heading{Examples}\vspace{0.3cm}
  	\begin{itemize}\setlength\itemsep{1em}
    \item Weather forecasting
  	\item Climate science
  	\item Engineering
  	\end{itemize}

  	\pause
    \break
    \heading{Toolkits}\vspace{0.3cm}
  	\begin{itemize}\setlength\itemsep{1em}
    \item Textbooks
  	\item High-performance computing\pause
  	\item {\color{red} Statistical decision theory}
  	\end{itemize}
  	\end{multicols}



\end{frame}
%---------------------------------------------------------------------------------------------------
%---------------------------------------------------------------------------------------------------
\begin{frame}{As-if analysis}\vspace{0.3cm}

In economics, however, we use the point estimates as a plug-in for the true parameter and the model is analyzed as-if the true parameters are known \citep{Manski.2021}.\vspace{0.3cm}
\pause
  	\heading{Consequences}
    \vspace{0.3cm}
  	\begin{itemize}\setlength\itemsep{1em}
  	\item Fragile findings as facts
  	\item Dueling certitudes stifle constructive debate
  	\item Knowledge gaps are not identified\pause
    \item {\color{red} Policy advice not framed as a decision problem under uncertainty}
  	\end{itemize}

    {\vspace{7.5pt} \hyperlink{Examples of as-if analysis}{\beamerbutton{Examples of as-if analysis}}}

\end{frame}
%---------------------------------------------------------------------------------------------------
%---------------------------------------------------------------------------------------------------
\begin{frame}{Contributions}\vspace{0.3cm}
  	\begin{itemize}\setlength\itemsep{1em}
    	\item We develop an approach that deals with parametric uncertainty and frames model-informed policy-making as a decision
problem under uncertainty.
    \pause
    \item We use the seminal human capital investment model by \citet{Keane.1997} as a well-known, influential, and empirically-grounded test case.
    \pause
    \item We document considerable uncertainty in their policy predictions and highlight the resulting policy recommendations from using different formal rules on decision-making under uncertainty.
  	\end{itemize}
\end{frame}
%---------------------------------------------------------------------------------------------------
%---------------------------------------------------------------------------------------------------
\begin{frame}{Roadmap}
\vspace{1cm}
\tableofcontents
\end{frame}
%---------------------------------------------------------------------------------------------------
%---------------------------------------------------------------------------------------------------
