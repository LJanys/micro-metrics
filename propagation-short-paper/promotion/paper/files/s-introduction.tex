%!TEX root = ../main.tex

\setcounter{page}{1}
%---------------------------------------------------------------------------------------------------
\FloatBarrier\section{Introduction}
%---------------------------------------------------------------------------------------------------
Structural microeconometricians use highly parameterized computational models to investigate economic mechanisms, predict the impact of proposed policies, and inform optimal policy-making \citep{Wolpin.2013}. These models represent deep structural relationships of theoretical economic models invariant to policy changes \citep{Hood.1953}. The sources of uncertainty in such an analysis are ubiquitous \citep{Saltelli.2020}. For example, models are often misspecified, there are numerical approximation errors in their implementation, and model parameters are uncertain. Therefore, most disciplines require a proper account of uncertainty before using computational models to inform decision-making \citep{Council.2012,SAPEA.2019}.\\

\noindent The following study focuses on parametric uncertainty in structural microeconometric models that are estimated on observed data. Researchers often do not account for parametric uncertainty and conduct an as-if analysis in which the point estimates serve as a stand-in for the true model parameters. They then continue to study the implications of their models at the point estimates \citep{Adda.2017,Blundell.2016,Eckstein.2019,Eisenhauer.2015b} and rank competing policy proposals based on the point predictions alone \citep{Blundell.2012,Cunha.2010,Gayle.2019,Todd.2006}. In fact, \citet{Keane.2011d} states in their handbook article that they are unaware of any applied work that reports the distribution of policy predictions under parametric uncertainty. To the best of our knowledge, this statement remains true more than a decade later. Consequently, economists risk accepting fragile findings as facts, ignoring the trade-off between model complexity and prediction uncertainty, and neglecting to frame policy advice as a decision problem under uncertainty.\\

\noindent To mitigate these shortcomings, we develop an approach that copes with parametric uncertainty in structural microeconometric models and embeds model-informed policy-making in a decision-theoretic framework. Ideally, policy-makers fix the parameter space ex-ante and then evaluate the policy options according to decision rules. However, this approach is often computationally intractable. We, therefore, follow \citet{Manski.2021}'s suggestion and, instead of using the parameter estimates as-if they were true, incorporate uncertainty in the analysis by treating the estimated confidence set as-if it is correct. We use the confidence set to construct an uncertainty set that is anchored in empirical estimates, statistically meaningful, and computationally tractable \citep{Ben-Tal.2013}. Instead of just focusing on the point estimates, we evaluate counterfactual policies based on all parametrizations within the uncertainty set.\\

\noindent We draw on statistical decision theory \citep{Manski.2013} to deal with the uncertainty in counterfactual predictions. This approach promotes a well-reasoned and transparent policy process. Before a decision, it clarifies trade-offs between choices \citep{Gilboa.2018}. Afterward, decision-theoretic principles allow constituents to scrutinize the coherence of choices \citep{Gilboa.2021}, ease the ex-post justification \citep{Berger.2021}, and facilitate the communication of uncertainty \citep{Manski.2019}.\\

\noindent We tailor our approach to the class of Eckstein-Keane-Wolpin (EKW) models \citep{Aguirregabiria.2010}. Labor economists often use EKW models to learn about human capital investment and consumption-saving decisions and predict the impact of proposed reforms to education policy and welfare programs \citep{Keane.2011d,Low.2017,Blundell.2017}. The analysis of these models poses serious computational challenges. During estimation, EKW models are solved thousands of times and even a single solution often takes several minutes. Thus, a decision-theoretic ex-ante analysis of alternative decision rules across the whole parameter space, as intended by \citep{Wald.1950}, is infeasible. Instead we construct an uncertainty set, a subset of the whole parameter space, and deal with the ex-post uncertainty after estimating the model. This compromise allows us to garner the benefits of using statistical decision theory to shape policy-making under uncertainty while ensuring the computational tractability of our analysis.\\

\noindent As an example of our approach, we analyze the seminal human capital investment model by  \citet{Keane.1997} as a well-known, empirically grounded, and computationally demanding test case. We follow the authors and estimate the model on the National Longitudinal Survey of Youth 1979 (NLSY79) \citep{NLSY.2019} using the original dataset and reproduce all core results. We revisit their predictions for the impact of a tuition subsidy on completed years of schooling. The economics of the model implies that the nonlinear mapping between the model parameters and predictions is truncated at zero, and we thus use the Confidence Set (CS) bootstrap \citep{Woutersen.2019} to estimate the confidence set for the counterfactuals. We document considerable uncertainty in the policy predictions and highlight the resulting policy recommendations from different formal rules on decision-making under uncertainty.\\

\noindent Our work extends existing research exploring the sensitivity of implications and predictions to parametric uncertainty in macroeconomics and climate economics. For example, \citet{Harenberg.2019} study uncertainty propagation and sensitivity analysis for a standard real business cycle model. \citet{Cai.2019} examine how uncertainties and risks in economic and climate systems affect the social cost of carbon. However, neither of them estimates their model on data. Instead, they rely on expert judgments to inform the degree of parametric uncertainty. They do not investigate the consequences of uncertainty for policy decisions in a decision-theoretic framework.\\

\noindent We complement a burgeoning literature on the sensitivity analysis of policy predictions in light of model or moment misspecification. For example, \citet{Andrews.2017} and \citet{Andrews.2020} treat the model specification as given and then analyze the sensitivity of the parameter estimates to the misspecification of the moments used for estimation. \citet{Christensen.2019} study global sensitivity of the model predictions to misspecification of the distribution of unobservables. \citet{Jorgensen.2021} provides a local measure for the sensitivity of counterfactuals to model parameters that are fixed before the estimation of the model.\footnote{For other examples, see \citet{Armstrong.2021}, \citet{Bonhomme.2021}, \citet{Bugni.2019}, and \citet{Mukhin.2018}.} This literature does not embed the counterfactual predictions in a decision-theoretic setting. Recent work by \citet{Kalouptsidi.2021a}, \citet{Kalouptsidi.2021}, and \citet{Norets.2014} studies (partial) identification and inference on counterfactuals. However, they all adopt the setup outlined in \citet{Rust.1987} and exploit the additive separability of the immediate utility function between observed and unobserved state variables, which does not apply to EKW models. In related work, \citet{Blesch.2021} conduct a decision-theoretic ex-ante analysis to determine optimal decision rules in \citet{Rust.1987}'s stochastic dynamic investment model where the decision-maker directly accounts for uncertainty in the model's transition dynamics. They only consider uncertainty in a subset of the model's parameters which are estimated outside the model and remain fixed to their point estimates during the analysis.\\

\noindent In Section \ref{Framework}, we describe the decision-theoretic framework for making model-informed decisions under parametric uncertainty using an illustrative example. After summarizing the empirical setting of \citet{Keane.1997} in Section \ref{Setting}, we present our results in Section \ref{Results}.  We complete our analysis in Section \ref{Conclusion} with a brief conclusion and outlook.
