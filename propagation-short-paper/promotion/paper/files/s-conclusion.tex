% !TEX root = ../main.tex
%---------------------------------------------------------------------------------------------------
\FloatBarrier\section{Conclusion}\label{Conclusion}
%---------------------------------------------------------------------------------------------------
We develop a generic approach that addresses parametric uncertainty when using models to inform policy-making. We propose a decision-theoretic analysis of computationally demanding structural models based on uncertainty sets. We construct the uncertainty sets from empirical estimates and ensure their computational tractability by using the confidence set bootstrap. We revisit the seminal work by \citet{Keane.1997} to document the empirical relevance of prediction uncertainty and showcase our analysis. Focusing on their ex-ante evaluation of a tuition subsidy, we report considerable uncertainty in the policy's impact on completed schooling. We show how a policy-maker's preferred policy depends on the choice of alternative formal rules for decision-making under uncertainty.\\

\noindent In our ongoing research, we pursue three avenues for further improvements. First, we link our work with the literature on inference under (local) model misspecification to refine the construction of our uncertainty sets. For example, \citet{Armstrong.2021} and \citet{Bonhomme.2021} propose different methods for taking misspecification into account when constructing confidence sets. Second, we incorporate ideas from the literature on global sensitivity analysis \citep{Razavi.2021} to identify the parameters most responsible for uncertainty in predictions. The attribution of importance based on Shapely values, familiar to economists from game theory, appears promising \citep{Owen.2014, Shapley.1953} as well. Third, we address our analysis's computational burden using surrogate modeling \citep{Forrester.2008}, which emulates the full model's behavior at a negligible cost per run and allows us to determine prediction uncertainty using a nonparametric bootstrap procedure.
